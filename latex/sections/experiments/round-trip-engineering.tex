\subsection{Round-trip experiments}
\subsubsection{Model consistency}
This experiment is targeting to Law 1 and Law 2 of the RTE meaning static consistency. UML models are inputs and used for generating code. Reversing engineering creates new models from the generated code. The new models are compared the original ones. Code generated from the new models is then also compared to that of the original ones. Execution time is also measured during reverse engineering to evaluate performance.

Each input model can contain many packages, classes, attributes, associations, dependencies, operation bodies, enumerations, data types that are transformed into C++ structs, default values. A model generator is developed to automatically randomly generate different UML models containing most of C++ features. The number of packages ranges from 1 to 15. Each package contains 100 classes, 10 enumerations and 10 data types. Class members including operations, operation bodies, attributes and operation parameters, and relationships between classes such as associations, generalizations and dependencies are randomly generated with an appearance probability for each kind of element. The appearance probabilities and information for setting up the model generator are shown in Table \ref{table:info-setup}. The resulting models are shown in Table. It is worth nothing that our generated code contains not only skeletons but also behavior code which is used to test the dependencies analysis. Furthermore, the generated code does not any strange comments to trace associated model elements. It looks like native code written by programmers.  

\begin{table}[]
\centering
\caption{appearance probabilities and information for setting up the model generator}
\label{table:info-setup}
\begin{tabular}{|l|l|}
\hline
Information description                                   & Value            \\ \hline
Number of packages                                        & 1-\textgreater15 \\ \hline
Number of classes per package                             & 100              \\ \hline
Number of attributes per class                            & 30               \\ \hline
Attribute appearance probability                          & 0.5              \\ \hline
Probability of being a pointer of an attribute            & 0.3              \\ \hline
Propability of typing an attribute a primitive type       & 0.4              \\ \hline
Propability of having a default value for an attribute    & 0.3              \\ \hline
Probability of being a reference of an attribute          & 0.3              \\ \hline
Number of operations per class                            & 30               \\ \hline
Probability of appearance for an operation                & 0.5              \\ \hline
Number of parameters per operation                        & 4                \\ \hline
Probability of appearance for a parameter                 & 0.5              \\ \hline
Probability of having a return parameter for an operation & 0.5              \\ \hline
Probability of being virtual for an operation             & 0.2              \\ \hline
Probability of being pure virtual for an operation        & 0.2              \\ \hline
Maximum number of generalizations                         & 300              \\ \hline
Number of C++ structs per package                         & 10               \\ \hline
Probability of appearance for a struct                    & 0.3              \\ \hline
Number of attributes per struct                           & 6                \\ \hline
Number of enumerations per package                        & 10               \\ \hline
Probability of appearance for an enumeration              & 0.3              \\ \hline
Number of enumerators per enumeration                     & 10               \\ \hline
Probability of appearance for an enumerator               & 0.5              \\ \hline
Number of function pointers per package                   & 10               \\ \hline
Probability of appearance for a function pointer          & 0.3              \\ \hline
\end{tabular}
\end{table}

Each class contains 24 operations, several generalization and associated associations. Table \ref{table:law1result} shows the statistics of models where the number of packages (P, PR), classes (C, CR), operations (O, OR), associations (A, AR), and generalizations (G, GR) of the input models and models create by reverse engineering, respectively. (a comparison for reverse rate + performance)


\begin{table*}[]
\centering
\caption{Input models and reversed models}
\label{table:law1result}
\begin{tabular}{|l|l|l|l|l|l|l|l|l|l|l|l|}
\hline
P  & PR & C    & CR & O     & OR & A    & AR & G   & GR & D & DR \\ \hline
1  &    & 25   &    & 600   &    & 38   &    & 20  &    & 0 &    \\ \hline
2  &    & 50   &    & 1200  &    & 76   &    & 40  &    & 0 &    \\ \hline
3  &    & 75   &    & 1800  &    & 114  &    & 60  &    & 0 &    \\ \hline
4  &    & 100  &    & 2400  &    & 152  &    & 80  &    & 0 &    \\ \hline
5  &    & 125  &    & 3000  &    & 190  &    & 100 &    & 0 &    \\ \hline
6  &    & 150  &    & 3600  &    & 228  &    & 120 &    & 0 &    \\ \hline
7  &    & 175  &    & 4200  &    & 266  &    & 140 &    & 0 &    \\ \hline
8  &    & 200  &    & 4800  &    & 304  &    & 160 &    & 0 &    \\ \hline
9  &    & 225  &    & 5400  &    & 342  &    & 180 &    & 0 &    \\ \hline
10 &    & 250  &    & 6000  &    & 380  &    & 200 &    & 0 &    \\ \hline
11 &    & 275  &    & 6600  &    & 418  &    & 220 &    & 0 &    \\ \hline
12 &    & 300  &    & 7200  &    & 456  &    & 240 &    & 0 &    \\ \hline
13 &    & 325  &    & 7800  &    & 494  &    & 260 &    & 0 &    \\ \hline
14 &    & 350  &    & 8400  &    & 532  &    & 280 &    & 0 &    \\ \hline
15 &    & 375  &    & 9000  &    & 570  &    & 300 &    & 0 &    \\ \hline
16 &    & 400  &    & 9600  &    & 608  &    & 320 &    & 0 &    \\ \hline
17 &    & 425  &    & 10200 &    & 646  &    & 340 &    & 0 &    \\ \hline
18 &    & 450  &    & 10800 &    & 684  &    & 360 &    & 0 &    \\ \hline
19 &    & 475  &    & 11400 &    & 722  &    & 380 &    & 0 &    \\ \hline
20 &    & 500  &    & 12000 &    & 760  &    & 400 &    & 0 &    \\ \hline
22 &    & 550  &    & 13200 &    & 836  &    & 440 &    & 0 &    \\ \hline
24 &    & 600  &    & 14400 &    & 912  &    & 480 &    & 0 &    \\ \hline
26 &    & 650  &    & 15600 &    & 988  &    & 520 &    & 0 &    \\ \hline
28 &    & 700  &    & 16800 &    & 1064 &    & 560 &    & 0 &    \\ \hline
32 &    & 800  &    & 19200 &    & 1216 &    & 640 &    & 0 &    \\ \hline
36 &    & 900  &    & 21600 &    & 1368 &    & 720 &    & 0 &    \\ \hline
40 &    & 1000 &    & 24000 &    & 1520 &    & 800 &    & 0 &    \\ \hline
\end{tabular}
\end{table*}
