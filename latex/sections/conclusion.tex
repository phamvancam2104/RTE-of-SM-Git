\section{Conclusion}
\label{sec:conclusion}
This paper presented a novel approach to round-trip engineering from UML state machines to code and back. The forward process of the approach is based on different patterns transforming UML state machine concepts such as states, transitions and events into an intermediate model containing UML classes. Object-oriented code is then generated from the intermediate model by existing code generators for programming languages such as C++ and JAVA. In the backward direction, code is analyzed and transformed into an intermediate whose format is close to the semantics of UML state machines. UML state machines are then straightforwardly constructed or updated from the intermediate format. 
 
The paper also showed the results of several experiments on different aspects of the proposed approach with the tooling prototype. Specifically, the experiments on the correctness of, the performance of, the semantic conformance of code generated by, and the cost of system development/maintenance using the proposed round-trip engineering are conducted. Although, the reverse direction only works if manual code is written following pre-defined patterns, the semantics of state machines is explicitly and intuitively present and easily to follow.

While the semantic conformance of code generated is critical, the paper only showed a lightweight experiment on this aspect. The reason is that the implementation of the prototype takes a lot of time. A systematic evaluation is therefore in future work. Furthermore, as evaluated in [7], the approach inheriting from the double-dispatch trades a reversible mapping for a slightly larger head. The reverse does not work concurrent state machine and several pseudo-states. Hence, future work should resolve these issues.
